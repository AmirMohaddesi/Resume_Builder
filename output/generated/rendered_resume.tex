\documentclass[11pt,a4paper]{resumecv}
\nopagenumbers
\usepackage{amssymb}

% If you really want the icons (\faMobile etc.), uncomment:
\usepackage{fontawesome5}
% --- Widen layout & fix ModernCV/resumecv mix ---
% Widen page margins (resumecv already loads geometry; override after class)
\geometry{margin=0.75in}

% Gentler lists and fewer breaks to pull content up
\usepackage{enumitem}
\setlist[itemize]{leftmargin=*, labelsep=0.5em, topsep=.15em, itemsep=.15em, parsep=0em}

% Ease line-breaking so LaTeX hyphenates less aggressively without overfull boxes
\sloppy
\emergencystretch=3em

\name{\textcolor{black}{Amirhosein}}{\textcolor{black}{Mohaddesi}}

% -------------------- custom helpers (unchanged) --------------------
\newcommand*{\customcventry}[7][.08em]{%
\begin{tabular}{@{}l}
{\bfseries #4}\\
{\itshape #3}
\end{tabular}
\hfill
\begin{tabular}{l@{}}
{\bfseries #5}\\
{\itshape #2}
\end{tabular}
\ifx&#7&%
\else\\[.15em]%
{\small #7}%  % no minipage: allows page breaks%
\fi
\par\addvspace{#1}}

\newcommand*{\awardentry}[3]{%
\begin{tabular}{@{}l}
#1
\end{tabular}
\hfill
\begin{tabular}{l@{}}
{\bfseries #2}\\
{\itshape #3}
\end{tabular}}

\begin{document}
\makecvtitle






\section{Summary}
AI/Robotics Engineer with a Ph.D. in Information \& Computer Science (Robotics focus) and 5–6 years of experience in ML, multi-agent systems, and simulation. Currently specializing in Agentic AI, integrating Large Language Models (LLMs) with ROS2 to enable autonomous, language-driven decision-making in robotics. Proficient in PyTorch, Hugging Face, and LangChain, with a strong track record in reinforcement learning, transformer models, and emerging agentic AI applications.

\section{Research Experience}

\pagebreak[2]
\customcventry
{Sep 2021 -- Jun 2025}
{\href{https://sites.socsci.uci.edu/~jkrichma/CARL/}{CARL Lab}, University of California, Irvine}
{Graduate Researcher}
{CA, USA}
{}
{%
\begin{itemize}[leftmargin=0.6cm, label={}]
  \item \textbf{Platform to train and study navigation agents in ROS2 environment, \textcolor{blue}{\href{https://github.com/AmirMohaddesi/Human-driven-navigation-strategies-in-a-ROS2-environment}{GitHub}}}
  \begin{itemize}[leftmargin=0.6cm, label={\textbullet}]
    \item Built a \textbf{multi-robot simulation platform in ROS2} to study human-inspired navigation strategies (\textit{Route, Survey, Mixed}) for team-based exploration and coordination.
    \item \textbf{Scaled the system from Webots to ROS2}, enabling larger swarm sizes, realistic sensor simulation, and integration with SLAM Toolbox, Nav2, and map\_merge.
    \item Conducted \textbf{simulation-based experiments} measuring task efficiency, coverage, and cooperation among robot teams with varying navigation strategies.
    \item Designed a \textbf{communication-aware coordination framework} allowing robots to share exploration goals and avoid redundant paths.
    \item Integrated \textbf{C++ and Python nodes} for navigation, mapping, and data logging; visualized multi-robot states in RViz for live analysis.
    \item Planned extension toward \textbf{imitation learning from human trajectory data} and future integration with \textbf{LLM-based reasoning} for agent communication and planning.
  \end{itemize}
  \item \textbf{Benefits of Varying Navigation Strategies in Robot Teams, \textcolor{blue}{\href{https://github.com/AmirMohaddesi/Benefits-of-Varying-Navigation-Strategies-in-Robots}{GitHub}}}
  \begin{itemize}[leftmargin=0.6cm, label={\textbullet}]
    \item Investigated the benefits of varying human-inspired navigation strategies (Route, Survey, Mixed) in robot teams through simulation-based experimentation.
    \item Developed a multi-agent setup in \textbf{Webots} using \textbf{Clearpath PR2} robots to evaluate task performance and strategy efficiency.
    \item Implemented obstacle avoidance and conflict resolution algorithms using a \textbf{C++} controller.
    \item Analyzed the impact of navigation strategy on task completion time, environment coverage, and coordination effectiveness.
    \item Demonstrated that mixed strategies yield a robust balance between exploration and efficiency in team-based navigation tasks.
    \item Contributed insights applicable to real-world exploration, search, and rescue missions involving autonomous robot teams.
  \end{itemize}
  \vspace{.25em}
  Mohaddesi, S.A.; Hegarty, M.; Chrastil, E.R.; Krichmar, J.L. \textit{“Benefit of Varying Navigation Strategies in Robot Teams.”} \textcolor{blue}{\href{https://link.springer.com/chapter/10.1007/978-3-031-71533-4_5}{Proceedings of SAB 2024}} (17th International Conference on Simulation of Adaptive Behavior), \textit{Lecture Notes in Computer Science}, Vol.~14993, pp.~63--77, Springer, 2024. DOI: \texttt{10.1007/978-3-031-71533-4\_5}. A related study was also published in \textcolor{blue}{\href{https://ieeexplore.ieee.org/abstract/document/10644592}{IEEE ICDL 2024}}.
  \item \textbf{Navigation and Cognitive Load in Telepresence Robots}
  \begin{itemize}[leftmargin=0.6cm, label={\textbullet}]
    \item Led a study evaluating cognitive load in manual vs.\ autonomous navigation using telepresence robots in a scavenger-hunt task.
    \item Developed autonomous navigation features, including real-time SLAM mapping in ROS and a custom GUI using PyQt.
    \item Designed experimental metrics for cognitive load, spatial awareness, user presence, and task efficiency.
    \item Conducted user studies to evaluate performance across autonomous and manual navigation modes.
    \item Concluded that autonomy reduced cognitive burden and improved movement efficiency, memory retention, and usability.
  \end{itemize}
  Pan, G.; Weiss, T.; Mohaddesi, S.A.; Szura, J.W.; Krichmar, J.L. \textit{“Navigation and Cognitive Load in Telepresence Robots.”} \textcolor{blue}{\href{https://ieeexplore.ieee.org/abstract/document/10644592}{IEEE ICDL 2024}}, pp.~1--6, DOI: \texttt{10.1109/ICDL61372.2024.10644592}.
\end{itemize}
}

\customcventry
{Jul 2020 -- Jul 2021}
{\href{https://nmi-lab.org}{NMI Lab}, University of California, Irvine}
{Assistant}
{CA, USA}
{}
{%
\begin{itemize}[leftmargin=0.6cm, label={}]
  \item \textbf{8-bit quantization technique for spiking neural networks}
  \begin{itemize}[leftmargin=0.6cm, label={\textbullet}]
    \item Developed an 8-bit quantized spiking neural network (SNN) using \textbf{PyTorch} for power-efficient embedded deployment.
    \item Introduced a custom quantization technique that reduced energy consumption by \textbf{12\%--18\%}.
    \item Maintained model accuracy within a \textbf{3\%--7\%} margin, validating efficiency/performance trade-offs.
  \end{itemize}
\end{itemize}
}

\section{Selected Projects}
\customcventry
{2025}{}{\textbf{Alter Ego: Personalized Conversational AI}, \textcolor{blue}{\href{https://huggingface.co/spaces/AMIXXM/Career_Conversation}{Hugging Face Demo}}}{}{}{% 
\begin{itemize}[leftmargin=0.6cm, label={\textbullet}]
  \item Designed and deployed a conversational AI agent that represents my professional background and research expertise.
  \item Built using \textbf{Python, Gradio, and the OpenAI API} with local persona embeddings for adaptive, context-aware conversations.
  \item Deployed on \textbf{Hugging Face Spaces} for public access and website integration.
\end{itemize}
}

\customcventry
{2024}{}{\textbf{Multi-Robot Coordination and Distributed Control System}, UC Irvine -- CARL Lab \textcolor{blue}{\href{https://github.com/AmirMohaddesi/Human-driven-navigation-strategies-in-a-ROS2-environment}{GitHub}}}{}{}{% 
\begin{itemize}[leftmargin=0.6cm, label={\textbullet}]
  \item Built a \textbf{ROS2 Humble-based simulation platform} for multi-robot coordination in disaster-response environments.
  \item Integrated \textbf{SLAM Toolbox, Nav2, map\_merge, and frontier exploration} for collaborative navigation and mapping.
\end{itemize}
}

\customcventry
{2022}{}{\textbf{Benefits of Varying Navigation Strategies in Teams of Robots}, UC Irvine -- CARL Lab \textcolor{blue}{\href{https://github.com/AmirMohaddesi/Benefits-of-Varying-Navigation-Strategies-in-Robots}{GitHub}}}{}{}{% 
\begin{itemize}[leftmargin=0.6cm, label={\textbullet}]
  \item Investigated how \textbf{Route, Survey, and Mixed} strategies affect multi-robot team efficiency in \textbf{ROS2 and Webots} with Clearpath PR2 robots.
\end{itemize}
}

\customcventry
{2021}{}{\textbf{Lunar Lander Trajectory Prediction (LLTP)}, UC Irvine -- PSYCH 239 \textcolor{blue}{\href{https://github.com/AmirMohaddesi/LLTP/tree/master}{GitHub}}}{}{}{% 
\begin{itemize}[leftmargin=0.6cm, label={\textbullet}]
  \item Built trajectory predictors with \textbf{RNNs} and \textbf{Convolutional Autoencoders} in \textbf{OpenAI Gym}; showed simple RNNs can learn Newtonian motion patterns.
\end{itemize}
}

\section{Skills}
\begin{itemize}[label=\textbullet]
\item \textbf{LLMs \& Agentic AI}: OpenAI API, LangChain, RAG, Prompt Engineering, LoRA/PEFT
\item \textbf{Machine Learning}: CNN, RNN, Autoencoders, Transformers, RL (DQN, PPO), VAE, SNNs, Quantization
\item \textbf{Programming}: Python, C++, Bash \ \ \textbf{Frameworks}: PyTorch, TensorFlow, OpenCV, HF Transformers, scikit-learn, Keras
\item \textbf{Data/Viz}: NumPy, Pandas, Matplotlib, Seaborn, SQL, Jupyter \ \ \textbf{Sim}: ROS2, Gazebo, Webots, OpenAI Gym
\item \textbf{OS/Tools}: Linux (Ubuntu), Windows, Git, Docker, Conda
\end{itemize}

\section{Education}
\customcventry
{Sep 2019 -- Jun 2025}
{\href{https://uci.edu/}{University of California, Irvine} \ \ Ph.D. in Information and Computer Science (ICS)}
{Irvine, CA}{}{}{%
\begin{itemize}[leftmargin=0.6cm, label={\textbullet}]
\item Cumulative GPA: 3.93
\item Relevant Courses: Embedded Ubiquitous Systems, Machine Learning, Neural Networks, Computational Neuroscience, Cognitive Robotics
\item Awards: Dean's Award, Donald Bren School of ICS
\end{itemize}
}

\customcventry
{Sep 2015 -- Jun 2019}
{\href{https://sharif.edu/}{Sharif University of Technology} \ \ B.S. in Computer Engineering}
{Tehran, Iran}{}{}{%
\begin{itemize}[leftmargin=0.6cm, label={\textbullet}]
\item Cumulative GPA: 3.95
\item Relevant Courses: Computer Architecture, Embedded Systems, Electrical Circuits, VLSI, Real-Time Processing
\end{itemize}
}

\section{Achievements}
\begin{itemize}[leftmargin=0.6cm, label={\textbullet}]
\item \awardentry{\textbf{Direct PhD fellowship} \ -- Donald Bren School of Information and Computer Science}{UCI, Irvine, CA}{2019}
\item \awardentry{\textbf{Dean's Award} \ -- Donald Bren School of Information and Computer Science}{UCI, Irvine, CA}{2019}
\item \awardentry{\textbf{Silver Medal} \ -- Iran's National Physics Olympiad, NODET}{Tehran, Iran}{2013}
\end{itemize}

\section*{Additional Info}
Authorized to work in the U.S. under F-1 OPT (valid through June 2026); Extendable by another two years under OPT-STEM. No immediate sponsorship required.

\end{document}